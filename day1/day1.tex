\documentclass[12pt]{article}
\usepackage{template}
\usepackage{listings}
\usepackage{xcolor}
\usepackage{ulem}
\usepackage{soul}
\usepackage{comment}
\usepackage{graphicx} % needed for \includegraphics

% Set Libertinus Serif font if using XeLaTeX/LuaLaTeX
\usepackage{iftex}
\ifXeTeX
    \usepackage{fontspec}
    \setmainfont{Libertinus Serif}
\else\ifLuaTeX
    \usepackage{fontspec}
    \setmainfont{Libertinus Serif}
\fi\fi

% Cover page information
\coverTitle{Day 1: Introduction to Robots and Robotics}
\coverAuthor{PRASAMSHA ADHIKARI}
\coverRollno{PUR080BCT060}
\coverTo{BIPIN SHRESTHA}
\coverLogo{robo.png}
\coverDate{2025/06/30}

% Code listing setup
\definecolor{codegreen}{rgb}{0,0.6,0}
\definecolor{codegray}{rgb}{0.5,0.5,0.5}
\definecolor{codepurple}{rgb}{0.58,0,0.82}
\definecolor{backcolour}{rgb}{0.95,0.95,0.92}

\lstdefinestyle{mystyle}{
    commentstyle=\color{codegreen},
    keywordstyle=\color{magenta},
    numberstyle=\tiny\color{codegray},
    stringstyle=\color{codepurple},
    basicstyle=\ttfamily\footnotesize,
    breakatwhitespace=false,         
    breaklines=true,                 
    captionpos=b,                    
    keepspaces=true,                 
    showspaces=false,                
    showstringspaces=false,
    showtabs=false,                  
    tabsize=2
}
\lstset{style=mystyle}

\begin{document}

% Generate cover page
\makecover

% Setup main document formatting
\setupmain

%\section{Heading}
%\subsection{Second Heading}

%This is regular text.

%\begin{lstlisting}[language=C++]
%# This is a code block.
%print("Hello")
%\end{lstlisting}

%\subsubsection{Third Heading}

%\begin{itemize}
%    \item This is an unnumbered list.
%    \item This is also an unnumbered list.
%\end{itemize}

\begin{comment}
\begin{enumerate}
    \item This is a numbered list.
    \item This is also a numbered list.
\end{enumerate}
\end{comment}

\begin{figure}[h]
    \centering
    \includegraphics[width=0.2\textwidth]{logo.png} % Reduced from 0.5 to 0.2
    \caption{TU IOE}
\end{figure}

\begin{comment}
\begin{table}[h]
    \centering
    \begin{tabular}{|c|c|}
        \hline
        One & Plan \\
        \hline
        Two & Draft \\
        \hline
        Three & Document \\
        \hline
    \end{tabular}
    \caption{Sample Table}
\end{table}

You can do almost anything with this typesetting system.

\textbf{Bold}\\
\textit{Italics}\\
\underline{Underline}\\
\sout{Strike}
\end{comment}



\vspace{2em}
{\huge \textbf{About the Training:}}
\vspace{1em}

Techmorph organized by Robotics Club, is 8 days long learning session. First day was about theory and basics about Robots and Robotics. The session was led by Binam Shrestha, a passionate and dedicated robotics learner an aspirant, a CEO of UTHHAN Robotics lab.
He shared his personal journey and valueable insights from his experience in the robotics field and his struggles. 

\vspace{2em}
{\huge \textbf{What We Learnt:}}
\vspace{1em}
\begin{enumerate}
    \item Introduction to Robots and Robotics
    \item History of Robots and their evolution
    \item Types of Robots (Autonomous, Humanoid, Augmented and Teleoperated)
\end{enumerate}

\vspace{2em}
{\huge \textbf{Other Tools and platforms }}
\vspace{1em}
\begin{enumerate}
    \item Familiar with Git and GitHub 
    \item Documentation tools like LaTeX and Typst
\end{enumerate}
\begin{comment}
\begin{center}
    \underline{\sout{\textbf{\textit{Everything all at once}}}} Center Aligned
\end{center}
\end{comment}
\end{document}
